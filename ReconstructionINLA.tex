\documentclass[12pt]{amsart}
\usepackage{amsmath,amsfonts,amssymb,amsthm,verbatim,multirow,url,subfig,footnote,graphicx,array,xr,booktabs,placeins,float,setspace}
\usepackage[usenames,dvipsnames]{color}
%  \usepackage[style=numeric,
%  doi=false,
%  isbn=false,
%  url=false]{biblatex}
\usepackage[utf8]{inputenc}
\usepackage[T1]{fontenc}
\usepackage[version=3]{mhchem}
\usepackage[compress,authoryear]{natbib}
\usepackage[top=1in, bottom=1in, left=1in, right=1in]{geometry}


\theoremstyle{plain}
\newtheorem{lema}{Lemma}
\newtheorem{coro}{Corollary}
\newtheorem{teo}{Theorem}
\newtheorem{prop}{Proposition}
\theoremstyle{definition}
\newtheorem{ap}{Assumption}
\newtheorem{defin}{Definition}
\theoremstyle{remark}
\newtheorem*{rk}{Remark}
\newcommand{\nn}{\mathbf}
\newcommand{\nns}{\boldsymbol}
\newcommand{\Hcal}{\mathcal H}

\newcommand{\bl}[1]{\color{ForestGreen}\textbf{[Bo: #1]}\normalcolor}
\newcommand{\lb}[1]{\color{MidnightBlue}\textbf{[LB: #1]}\normalcolor}
\newcommand{\jeg}[1]{\color{ProcessBlue}\textbf{[JEG: #1]}\normalcolor}



\begin{document}
\doublespacing
\title[]{Efficient Reconstructions of Common Era Climate via Integrated Nested Laplace Approximations.}

\author[]{Luis A. Barboza}
\address{Centro de Investigacion en Matematica Pura y Aplicada (CIMPA)-Escuela
  de Matematica, Universidad de Costa Rica\\
San Jos\'e, Costa Rica}
\email{luisalberto.barboza@ucr.ac.cr}


\author[]{Julien Emile-Geay}
\address{Department of Earth Sciences \\
  University of Southern California \\
  Los Angeles, California, USA.
}
\email{julieneg@usc.edu}


\author[]{Bo Li}
\address{Department of Statistics \\
  University of Illinois at Urbana-Champaign \\
  Champaign, Illinois, USA.
}
\email{libo@illinois.edu}

\author[]{Wan He}

\date{\today}
\keywords{Hierarchical Bayesian Model, INLA, Paleoclimate Reconstruction}
\subjclass[2010]{}
\maketitle

\begin{abstract}
Paleoclimate reconstruction on the Common Era (1-2000AD) provide critical context for recent warming trends. This work leverages integrated nested Laplace approximations (INLA) to conduct inference under a Bayesian Hierarchical Model using data from three sources: a state-of-the-art prox database (PAGES 2k), surface temperature data (HadCRUT4), and latest estimates of external forcing data from several sources. 
INLA's computational efficiency allows to explore several model formulations (without forcings, with forcings, and forcings + internal variability), as well as five data reduction techniques.  Two different validation exercises find a small impact of data reduction choices, but a large impact for model choice, with best results for the two models that incorporate external forcings. These models confirm that man-made greenhouse gas emissions are the largest contributor to temperature variability over the Common Era, followed by volcanic forcing. Solar effects are indistinguishable from zero.  INLA provides a fast way to estimates the posterior mean, comparable with the much costlier Monte Carlo Markov Chain procedure, but with wider uncertainty bounds. We recommend using it for exploration of model designs, but full MCMC solutions should be used for uncertainty quantification.  
\end{abstract}

\section{Introduction.}
\label{sec:intro}

Earth's climate presents a continuum of variability, with periodic and non-periodic fluctuations ranging from 1 to $10^8$ years \citep{pelletier_power_1998}. In particular, variability on scales of decades to centuries is of paramount importance for adaption and planning to anthropogenic climate change, yet is incompletely sampled by the relatively short historical record of wide-spread instrumental observations, going back to about 1850 CE \citep{AR5_chap5}. It is thus critical to reconstruct these variations from the paleoclimate record as quantitatively as possible.  A particular focus has been reconstructions of global or hemispheric temperature from high-resolution proxy observations \citep{Jones_Holocene09}.

Many methods have been developed to reconstruct past climates, particularly temperatures:  principal component regression
\citep{MBH98,luterbacher2004european}, regularized forms of the expectation-maximization algorithm \citep{Schneider2001,mann2007robust,JEG10a,Guillot_AOAS2015}, canonical
correlation analysis \citep{smerdon2010pseudoproxy,Wang_CP2014}, pairwise comparison
\citep{Hanhijarvi2013}, data assimilation \citep{Lee_CD08,Hakim2016}, and Bayesian hierarchical models (BHMs). 

BHMs offer several distinct advantages for paleoclimatic reconstruction. They can (i) treat different sources of uncertainty in a natural way, (ii) incorporate prior knowledge of the system in a logically-coherent manner, and (iii) allow for both inference and prediction \citep{Tingley_QSR2012}. 
Many studies have employed BHMs \citep[e.g.][hereafter, "B14"]{boli1, tingley2013_Ext,Barboza2014}); some have used space-state
schemes to linearly relate information that comes from paleoclimate observations together with information about external climatic forcings, resulting from 
well-mixed greenhouse gases, volcanic activity, and variations in solar output. Three problems arise in this
case: the need to reduce dimensionality, the complexity/realism of the model, and the execution time of the numerical procedure. 

B14 proposed a method that jointly models the variability of the
temperature series (as a latent process) as well as the variability of those
climatic and biological observations that serve as approximations of this process (``proxies'').
The authors found that long memory error terms are necessary in absence of
external forcing information within a BHM, but the
presence of external forcing information substantially improves the reconstruction. The scope of the study was restricted by the computational requirements of the Monte Carlo Markov Chain (MCMC) procedure, which limited the sensitivity analysis and forced severe levels of data reduction. 

In this article we extend the work of B14 by leveraging Integrated Nested Laplace
Approximations (INLA). Doing so lightens the computational burden, which allows to (a) explore new model designs inspired by the physics of the problem; (b) consider various choices for data reduction; and (c) take the non-stationary nature of the observational network into account.  In addition, this work makes use of the latest estimates of radiative forcing, as well as a state-of-the-art, open-access compilation of community-curated paleoclimate observations \citep{PAGES2kSD2017short}. This ensures that our calculations are using the best available data and are completely reproducible.  

The article is organized as follows: we start by describing the datasets (section 2), and the methodology (section 3). Results are then presented in section 4, discussed in Section 5, before concluding in section 6. 

\section{Datasets.}
\label{sec:data}

\subsection{Proxy data.}
Reconstructions of past climates rely on ``proxies'': indirect observations of climate, as recorded in borehole, coral, documentary, glacier
ice, lake and marine sediment, sclerosponge, speleothem and tree-ring archives
\citep{Jones_Holocene09}. The PAGES2k global multiproxy database is a community-driven effort to
synthesize all publicly-archived, temperature-sensitive proxy records of the
past 2,000 years \citep{PAGES2kSD2017short}. The most recent effort gathers 692 records from 648 locations, including all continental regions and major ocean basins. The records are from trees, ice, marine and lake sediments, corals, speleothems, and documentary evidence. They range in length from 50 to 2000 years, with a median of 547 years, while temporal resolution ranges from biweekly to centennial.
  The vast majority of records are annually-resolved, with minimal dating uncertainty. Here, the data used have been mapped to a standard normal using the method of \cite{vanAlbada2007}. The data are available in a standard format (LiPD) readable in R, Python and Matlab, to ensure reproducible workflows \citep{lipd_cp}.   

 Each of those proxies has different time horizons (Fig.~\ref{fig:proxy}, bottom), which creates challenges for inference. Unlike previous studies (e.g. B14), we strive to take into account the information available in most proxies, despite their temporal diversity.
In order to select proxies with high predictive power, we first chose those series with large correlations with respect to
their closest spatial temperature record in the HadCRUT4.2 dataset \citep{Morice2012}. More
details on this ``screening'' procedure, which controls for the multiple test problem \citep{BenjaminiHochberg95}, can be found in \citet{PAGES2kSD2017short}; it whittles down the database to 257 proxies (Fig.~\ref{fig:proxy}).   
\begin{figure}
  \centering
  \includegraphics[scale=0.40]{CombinedMap_Area}
  \caption{Distribution and temporal availability of PAGES2k proxies, after applying the screening procedure \cite{PAGES2kSD2017short}.}
  \label{fig:proxy}
\end{figure}
%As mentioned, proxies have different temporal horizons, as it can be shown in thesecond panel of Figure~\ref{fig:proxy}. 

\begin{table}[h!]
  \centering
  \begin{tabular}{c|c|c}
    \toprule
    Group & Interval (year AD) & Number of Proxies\\
    \midrule
    1 & 1-250 & 19 \\
    2 & 251-500 & 25 \\
    3 & 501-750 & 29 \\
    4 & 751-1000 & 33 \\
    5 & 1001-1250 & 54 \\
    6 & 1251-1500 & 65 \\
    7 & 1501-1750 & 105 \\
    8 & 1751-2000 & 146 \\
    \bottomrule
  \end{tabular}
  \caption{Distribution of proxies according to their temporal availability.}
  \label{tab:distdate}
\end{table}

\vskip -0.7cm

Due to the diversity of start dates in the proxies database (Fig.~\ref{fig:proxy}), we gather proxies into non-homogeneous groups where each group has temporal availability within a 250y interval (Table \ref{tab:distdate}).  Proxy series whose proportion of missing annual observations is larger than 5\% during the calibration period (1900-2000, as in  B14) are excluded.  %\jeg{This gives me pause. Did you exclude all series lacking >5\% of data in a given 250y nest, or overall. If overall, you are missing out on a LOT of proxies!!! We should redo the calculations with aless restrictive threshold}\lb{The percentage of missing observations is computed over the calibration period (1900-2000), so we don't loose too much information.}. 


\subsection{Temperature data.}
We estimate Global Mean Surface Temperature (GMST) from the HadCRUT4 global temperature dataset provided by the Met Office Hadley
Centre and the Climatic Research Unit at the University of East Anglia, UK
(version 4.4.0.0). The dataset consists of instrumental, {\it in situ}
observations of surface temperature over land \citep{Jones2012} and ocean
\citep{Kennedy2011a,Kennedy2011}. The
observations are expressed as anomalies relative to the monthly-mean seasonal cycle over the 1961-1990 period, in degrees Celsius. Though HadCRUT4 features a rather sophisticated analysis of error sources \citep{Morice2012}, we neglect these uncertainties against the much larger uncertainties affecting paleoclimate observations, and simply use the median estimate, averaged on an annual basis.
%\jeg{Luis, are you using the GMST provided by HadCRUT4, or do you calculate your own from the spatial data?}\lb{I'm using the file  HadCRUT.4.4.0.0.ama\_ns\_avg\_1850-2015.txt that you provided me.} % thanks for the clarification. 

%\jeg{Note: it's a little too late for that, but one day it would be nice to incorporate the uncertainty in instrumental temperatures into your analysis. Perhaps mention this in the discussion as a possible application of INLA?}\lb{Yes, I will include that in the discussion.}

\subsection{Forcing data.}
These use the most recent compilations from the PMIP4-CMIP6 project \citep{JungclausGMD17}: 

\noindent {\bf Volcanic forcing} from the evolv2k dataset
  \citep{Toohey2016}, which reconstructed zonal mean aerosol optical depth at 550 nm, covering the 500 BCE to 1900 CE time period. For 1900 (or 1850) to present, \cite{Thomason2016} is
  used to fill in the forcing table. The data were given as a function of latitude, so were area-weighted and averaged to form a global estimate (Fig.~\ref{fig:forcings}, top).
  
\noindent {\bf Solar forcing} data is computed from the SATIRE-H
   dataset \citep{Vieira2011}. Irradiance is
  provided on a decadal basis from 9495BC - 1939AD and then on a daily basis
  from 1940AD onwards. We interpolated the data at annual resolution using BSplines (Fig.~\ref{fig:forcings}, middle). 
  
\noindent {\bf Greenhouse-Gases concentrations}: hemispheric means of mole
  fraction of carbon dioxide in air (ppm) with annual
  resolution \citep{Meinshausen_GMD2017} as well as ice core measurements prior to that date \citep{JungclausGMD17} (Fig.~\ref{fig:forcings}, bottom).


\begin{figure}
  \centering
  \includegraphics[scale=0.40]{forcings}
  \caption{Main climate Forcings of the Common Era (1-2000 AD): volcanic, solar, and carbon dioxide. AOD = Aerosol optical depth. TSI = total solar irradiance. ppm = parts per million. See text for details.}
  \label{fig:forcings}
\end{figure}

\section{Methodology}\label{sec:model}
\subsection{Data reduction methods}
\label{sec:rp}

For some intervals in Table \ref{tab:distdate}, the proxy data matrix is large compared to the 101 yearly samples used to train the model (1900-2000). This ``large $p$, small $n$'' problem calls for some form of data reduction. Following B14, we generate a set of ``Reduced Proxy'' ($RP$) time series, which condense individual proxy time series into a single time series with larger predictive power over the GMST target. Since this is a critical choice to make, we extend B14 by carefully investigating five common data reduction methods:

\jeg{You want to say something about screened vs unscreened here?}


%An important aspect we notice from the distribution of Table \ref{tab:distdate} is that the number of proxies can be greater or  close to the number of available observations in the calibration period defined as 1900-2000CE (101 observations), which can cause overfitting issues or dimensionality problems with classical linear models. . For the above reasons, we investigated various data reduction techniques with the aim of establishing  a reliable linear model between the observed anomalies and the corresponding proxies in Table \ref{tab:distdate}, based on the data in the calibration period. Many different data reduction methods have been developed; we focus on five popular ones:  
%(Description of the five methods including the basic idea, the unique feature, the pros and cons of the method)
%\begin{description}

{\bf Lasso Regression (LR)}
  The Lasso regression penalizes the usual sum of squares with an argument
 containing the sum of the absolute values of each coefficient in the classical
 linear regression model, multiplied by an additional smoothing parameter \citep{Tibshirani1996}. Due
 to the geometric nature of the term of penalization, the search of estimators
 tends to assign values very close to zero to variables that have almost null
 effects with respect to the dependent variable, which makes the resulting
 models easily interpretable. This method is very common to data reduction and easy to implement, but it often tends to select overly complex models, that is,
 it tends to show "false positives" in the variable selection process 
 \citep{Fan2010}. The Lasso may also run into inconsistency issues when the
 variables are highly correlated \citep{Zou2005}.
 We used 10-fold cross validation to select the smoothing parameter when we carried out Lasso regression (see \cite{Tibshirani1996} for more details). 
 
{\bf Sparse Partial Least Squares (sPLS)}
  Partial least squares seek to reduce the high-dimensionality issues of the
  design matrix in  
  linear regression models through a latent matrix whose columns maximize
  the product of the linear correlation between predictors and responses and the
  variance of responses. The sPLS method further introduces sparseness to the partial least squares
  estimators by means of a $L_1$-penalty with a thresholding parameter, in order to avoid inconsistencies when there is a
  substantial number of noisy covariates \citep{Chun2010,Chung2013}. However, this method is inefficient in
  measuring the statistical significance of whether the parameters associated with certain
  variables are effectively zero \citep{OlsonHunt2014}. In our implementation, the
  thresholding parameter involved in sPLS is estimated using a 10-fold cross-validation criteria.    

{\bf Sliced Inverse Regression (SIR) with CSS selection}
  In general, SIR methods \cite{Li1991}  reduce the excess dimension in a non-parametric setting through the
  estimation of the linear space spanned by the coefficients of the covariables,
  also known as \textit{effective dimension reduction} (EDR) subspace. 
  This subspace is obtained by an approximate eigenvalue decomposition that
  involves an estimation of the covariance matrices of the design matrix and
  the conditional expectation of the explanatory variables given the
  response. The
  estimation of the covariance matrix requires to partition the dependent variable into subgroups, called \textit{slices}.
  The SIR method can capture both linear or nonlinear associations between the response and
  covariates. However, the estimation of the dimension
  reduction space does not actually lead to a variable selection procedure and the
  covariance estimation relies heavily on the homogeneity of the response within
  each slice \cite{Wu2010}. Because of this, we opt to incorporate the CSS
  (\textit{Closest submodel selection}) variable 
  selection procedure into the SIR method. Furthermore, to better deal with the fact that there is a larger number of
  covariates than observations, we employed the SIR-QZ algorithm, an upgrade of the SIR method based on the generalized Shur decomposition for underdetermined cases \citep{Coudret2014,Coudret2017}.   
  %\footnote{The SIR-QZ method is based on the generalized Shur decomposition and it aims to solve the eigenvalue problem of SIR for undetermined cases.} instead  (see \cite{Coudret2014} and \cite{Coudret2017}).
  %This compound method allows us to select
  %reduced models that are close to the complete model (possibly with excess
  %of dimension) in terms of their eigenvalue decompositions. 
  Finally, we studied the association between proxies and temperatures
  through a linear regression between the observed anomalies and a number of EDR directions, i.e. an orthogonal basis of the EDR subspace, determined by marginal dimension tests \citep{Cook2004}. 
  
{\bf Principal Component Regression (PCR)}
PCR simply means that we replace the original covariates by their PCs. To select how many and which PCs to use, we fitted a regression model between the temperature and PCs of eight different sets of covariates selected in each of the eight nests (Table 1) based on the training data from the calibration period. We then selected the number of PCs in each of
the eight regressions as the minimum number that attains, for the first time, an adjusted
$R^2$ of at least 70\% in each case. PCR is often used when covariates are
highly-correlated or when the number of covariates is larger than the number of 
 observations. A caveat of this method is that the principal components
with smaller contribution to variance are not necessarily the ones that
associate less with the dependent variable in a linear regression model 
\citep{Jolliffe1982,Tibshirani1996}. 

{\bf Supervised Principal Components (sPCR)}
Because of the above-mentioned caveat of directly using PCR, \cite{Bair2006} developed a technique where PCR is applied only
to a certain subset of covariates that exhibits a considerable amount of association
 with the dependent variable, and the threshold of ``considerable" is chosen through
cross-validation. Compared to PCR, the sPCR ensures the dimensionality
reduction on the covariates space while taking the association between
the covariates and the dependent variable into account. In general, its performance is quite similar to sPLS  \citep{Chung2013}.  
%\end{description}

\begin{figure}[h!]
  \centering
 \includegraphics[scale=0.40]{RPs_type} 
  \caption{Reduced Proxies among methods.}
  \label{fig:RPs}
\end{figure}

The data reduction allows us to fit linear regression models between
temperatures and proxies. After we fit a linear regression model under each of
the five data reduction methods and for each of the eight proxy groups listed in
Table \ref{tab:distdate}, we compute a single reduced proxy series following
B14 for each group. All reduced proxies are shown in Fig.~\ref{fig:RPs}.  These series are highly correlated, with the reduced proxies obtained by PCR standing out as least similar to the others by this metric (Fig.~\ref{fig:CorrRPs}).

\begin{figure}[h!]
  \vskip -1cm
  \centering
 \includegraphics[scale=0.35]{CorMatrixRPs} 
  \vskip -1.1cm \caption{Correlation Matrices among 5 different Reduced Proxy series}
  \label{fig:CorrRPs}
\end{figure}


\subsection{Model Specification}
\label{sec:modelspec}
 The first level of a BHM always models the likelihood of the data \citep{Tingley_QSR2012}. The second level models the temperature process, and the third models observational uncertainties. 
 
As in  \citet{boli1} and B14, our process model includes radiative forcings, since they are known to drive the temperature process. This also allows to {\it
  attribute} forcings \citep[i.e., determine causality,][]{HegerlZwiers:2011} as
part of the inference procedure. However, this raises the spectre of
overfitting, whereby the model would discount the noisy proxy data and place
undue weight on the forcings, and not enough on internal variability. To
mitigate this risk, it may be preferable to model temperature fluctuations as a
smooth function of time (say, via splines) without including forcings, then
perform forcing attribution on the inferred temperature posterior
\citep{Schurer2013a, Schurer2013b}, which guarantees independence: this way, if the reconstruction bears a strong resemblance to the forcings, it is only because the latter are reflected in the values of the predictors, not because they were fed to the model. In this section we explore both end-members, as well as an intermediate case. We first define:
%Below we first define notation:
%Our models are based on the one presented in B14. Now we define some notation, %assuming that all the variables are set at time $t$.
\vspace{-.3cm}
\begin{itemize}
\item $RP_t^i$: $i$-th reduced proxy at time $t$.
  
\item $T_t$: temperature anomaly at time $t$.
  
\item $\tilde C_t = \log (C_t)$: Transformed greenhouse gases. The log
  transformation is chosen to approximate the radiative forcing due to changes
  in the equivalent CO$_2$ concentration (see B14).
  
\item $\tilde V_t = \log (-V_t+1)$: Transformed volcanic forcing. See more details in B14 for the choice of the transformation.
  
\item $B_t^{k,\tau}$: $k$-th B-Spline basis function at time $t$ with a uniform knot
  sequence $\tau$ \citep{DeBoor2001,Ramsay2005}. Here we choose
  cubic B-spline bases and we denote $K(\tau)$ as the total number of basis elements.  
\end{itemize}
We then define the first level of BHM as $RP_t^i=\alpha_0^i+\alpha_1^iT_t+\epsilon^i_t$,  
where $\{\alpha^i_j\}$ are intercepts ($j=0$) and slopes ($j=1$) for $i=1,\ldots,N$,  and $\epsilon^i_t$ are normally-distributed random variables with finite variances
$\{\sigma^2_i\}$. Note that in our case $N=8$; the time variable $t$ is defined on each nest according to the
intervals of Table~\ref{tab:distdate}. We explore three formulations of the process level:

\noindent {\bf No forcing (model "NF")}
The main idea of this model is to provide a baseline that ignores external forcings. To capture low-frequency behavior, we include a smoothing function as: 
  \begin{align}\label{eq:M1}
    T_t=\beta_0+\sum_{k=1}^{K(\tau)}\beta_k B_t^{k,\tau}+\eta_t,
  \end{align}
where $\beta_k$ are coefficients for B-spline bases, and
$\eta_t$ are also normally-distributed random variables with finite variances
$\sigma^2_{\eta}$. For simplicity, the error terms $\epsilon^i_t$ and $\eta_t$
are assumed to be independent.  

\noindent {\bf With forcing (model "WF")}
Like B14, this model explicitly models temperature as linearly driven by radiative forcing:
  \begin{align}\label{eq:M2}
    T_t=\beta_0+\beta_1S_t+\beta_2\tilde V_t+\beta_3\tilde C_t+\eta_t.
  \end{align}
%Both first level and second level models are defined for $t=1,\cdots,2000$ (Common Era). 
Note that Models NF and WF simply assume an IID error structure. This is because B14 
found that complex error structures make little difference when forcings are added to the reconstruction.

\noindent {\bf "Mixed" Model} 
Finally we consider the more realistic case where temperature reflects both external forcings and internal dynamics. This model is a combination of equations \eqref{eq:M1} and \eqref{eq:M2}, as follows:
  \begin{align}\label{eq:M3}
    T_t=\beta_0+\beta_1S_t+\beta_2\tilde V_t+\beta_3\tilde C_t+\sum_{k=1}^{K(\tau)}\gamma_k B_t^{k,\tau}+\eta_t
  \end{align}
  where $\gamma_k$ are the coefficients for the B-Spline basis. %\lb{Julien: please include an explanation why this model is interesting in the climatology community.} \jeg{done}
  This last case is of most obvious relevance to climate dynamicists. The downside is that the model is more complex, thus making estimation more challenging. \jeg{true?}



\subsection{Computing posteriors with INLA}

The computational challenge of MCMC inference has been a concern for Bayesian paleoclimate reconstructions. It is crucial to overcome this  bottleneck before we can move forward to a more complex space-time reconstruction. Here we introduce the INLA sampling strategy to accelerate the MCMC procedure, as a proof of concept for more comprehensive models. The INLA approach is applicable to a general specification for which the mean $\eta_i$ of the observations $y_i$ follows a linear structure:

%The computation is slow for even a global reconstruction, let alone a space-time reconstruction in our mind. We now introduce INLA to our reconstruction by adapting our model to the general specification of  the INLA framework. Now it is fast and we plan to extend this to space-time reconstruction. 


\begin{align}\label{eq:meanINLA}
  \eta_i = \alpha +\sum_{m-=1}^M\beta_mx_{mi}+\sum_{l=1}^Lf_l(z_{li})
\end{align}
where $\alpha$ represents an intercept, the coefficients
$\mathbf{\beta} = (\beta_1,\ldots,\beta_M)$ relate $M$ covariates
$(x_1,\ldots,x_M)$ to $\eta_i$, and $f = \{f_1(\cdot),\ldots,f_L(\cdot)\}$ is a collection of
random effects defined on a set of $L$ covariates $(z_1,\ldots,z_L)$ (see
\cite{Rue2009} and \cite{Blangiardo2013}). 
Denote the set of random variables as
$\theta = (\alpha,\beta,f)$ with $K$ hyperparameters $\psi =
\{\psi_1,\ldots,\psi_K\}$, and the vector of observations as $y=(y_1,\ldots,y_n)$. Model \eqref{eq:meanINLA} leads to conditional independence of $y$ given $\theta$ and $\psi$:
\begin{align*}
  p(y|\theta,\psi)=\prod_{i=1}^np(y_i|\theta_i,\psi).
\end{align*}
In our models, if we consider $y_i$ as the reduced proxies,  $\eta_i$ the
linear mean of the reduced proxies, $x_m$ the external forcings and/or the set
of spline basis, and $f(z_l)$ the latent variables (temperature anomalies in
our case), then our models fall into the general specification of INLA. %\bl{Is the above correct? I thought $y_i$ is the observed temperature anomaly and $\eta_i$ the mean of temperature anomalies. If what you said is correct, then after you estimate the parameters, how do you sample the reconstructed temperatures?}
The main objectives of our Bayesian estimation are to compute the
marginal posterior distribution of each parameter in $\theta$:
\begin{align*}
  p(\theta_i|y) = \int p(\theta_i,\psi|y)d \psi = \int p(\theta_i|\psi,y)p(\psi|y)d \psi
\end{align*}
To attain computational advantages, INLA assumes that the prior of
vector $\theta$ is a multivariate normal random vector with a precision matrix
that depends on hyperparameters $\psi$. INLA further 
approximates the two components $p(\psi|y)$ and $p(\theta_i|\psi,y)$. The first component is replaced by its a
Laplace Approximation (see \cite{Tierney1986}):
\begin{align*}
  p(\psi|y)=\frac{p(\theta,\psi|y)}{p(\theta|\psi,y)}\propto \frac{p(\psi)p(\theta|\psi)p(y|\theta)}{p(\theta|\psi,y)}
            \approx  \frac{p(\psi)p(\theta|\psi)p(y|\theta)}{\tilde p(\theta|\psi,y)} \Biggm |_{\theta=\theta^*(\psi)} := \tilde p(\psi|y),
\end{align*}
where $\tilde p(\theta|\psi,y)$ is the Gaussian approximation of
$p(\theta|\psi,y)$ and $\theta^*(\psi)$ is its mode (see \cite{Rue2009}). The
second component can be approximated in a similar way:
\begin{equation}\label{eq:second}
  p(\theta_i|\psi,y)  =\frac{p((\theta_i,\theta_{-i})|\psi,y)}{p(\theta_{-i}|\theta_i,\psi,y)} %\notag 
   \approx \frac{p((\theta_i,\theta_{-i})|\psi,y)}{\tilde p(\theta_{-i}|\theta_i,\psi,y)} \Biggm |_{\theta_{-i}=\theta_{-i}^*(\theta_i,\psi)}:=\tilde p(\theta_i|\psi,y),
\end{equation}
where $\theta=(\theta_i,\theta_{-i})$, $\tilde p(\theta_{-i}|\theta_i,\psi,y)$ is the Gaussian approximation of
$p(\theta_{-i}|\theta_i,\psi,y)$ and $\theta_{-i}^*(\theta_i,\psi)$ is its mode,  
The approximation in \eqref{eq:second} possesses good precision, but it is very time demanding because it
requires to recompute $\tilde p(\theta_i|\psi,y)$ for each value of $\theta$ and $\psi$. A
more efficient approach is to use the simplified Laplace Approximation that is
based on a Taylor's expansion of $\tilde p(\theta_i|\psi,y)$ in 
\eqref{eq:second}. As mentioned in \cite{Rue2009} and \cite{Blangiardo2013},
INLA first explores the marginal joint posterior of the hyperparameters $\tilde
p(\psi | y)$ in order to locate the mode and then performs a grid search to 
produce a set of ``relevant'' points $\{\psi^*\}$ together with a set of weights
$w_{\psi^*}$ as an approximation of this marginal distribution. The marginals
$p(\psi^*|y)$ are then refined using interpolation methods. Finally, the marginals
$\tilde p(\theta_i|y)$ are obtained as follows:
\begin{align*}
  \tilde p(\theta_i|y) \approx \sum_{\psi^*}\tilde p(\theta_i|\psi^*,y)\tilde p(\psi^*|y)w_{\psi^*}.
\end{align*}

\subsection{Priors}
The models proposed in section \ref{sec:modelspec} were
implemented using the R package \textbf{r-inla} (www.r-inla.org). The
implementation followed the methods provided in \cite{Ruiz-Cardenas2012} and
\cite{Muff2015} on the use of the INLA methodology in state-space models,
dynamic linear models, and in general models whose mean can be written
into equation \eqref{eq:meanINLA}. These
methods together with \cite{Martins2013} allow INLA to be applicable to a wide array of Bayesian models such as BHM that consists of several layers. \jeg{the above paragraph reads a bit too much like advertising, and does not help the paper. I recommend leaving out the last sentence, if not more.}

Like all Bayesian procedures, INLA requires priors for unknown parameters, given below:
\begin{itemize}
\item $\alpha^i_j\sim N(0,3)$, $\beta_\ell \sim N(0,3)$ for $i=1,\ldots,N$, $j=0,1$, and  $\ell=0,\ldots,3$
  for Model WF and $\ell=0,\ldots,K(\tau)$ for Model NF. The choice of the variance is
  completely arbitrary, but the main idea is to select a relatively large one.
  
\item $\rho_i := -\log \sigma^2_{\epsilon^i}\sim \text{log-gamma}(1,10^{-20})$
  (very small precision) for $i=1,\ldots,N$.
  
\item $\rho_0 := -\log \sigma^2_\eta \sim \text{log-gamma}(1,10^{-20})$ (very
  small precision).
\end{itemize}


\section{Results.}
\label{sec:results}

In the following, the results use model WF for comparison with B14, unless otherwise specified.

%\jeg{Here the baseline seems to be WF. WOuld it make sense to choose "Mixed" or "NF" instead?} \lb{Our 2014's paper studies a WF model, so we can use WF as a baseline for comparison purposes with our previous work.} 

\subsection{Impact of data reduction choices}
As a first exercise, we analyze the change in the predictive capacity of the
 model when more equations involving proxies are included. We used
two proper scoring rules \cite{Gneiting2007a} as measures of predictive ability: the Interval Score at $\alpha$ level
(IS$_\alpha$) and the Continuous Ranked
Probability Score (CRPS). These scores have been previously employed in the
verification of point forecasts in environmental sciences for example, as well as the area
of paleoclimatic reconstructions (see B14 and
\cite{Scheuerer2014}). Table~\ref{tab:comparisontot} contains the predictive
measures using INLA's prediction intervals and the observed
  anomalies over 1850-1899 as an out-of-sample validation interval. We used INLA's availability of computing direct
posterior densities in order to compute interval scores and, in the CRPS case,
we used Monte Carlo samples obtained from INLA.  The
  last column consists of the mean square error (MSE) between the smoothed
  reconstruction and the borehole-based
reconstruction of \cite{Pollack2004} over 1600-1899 (PS04 dataset) as an out-of-sample
validation interval. \jeg{do you need me to better justify this dataset?}

Smoothing was accomplished through a Butterworth low-pass filter with a 50-year cutoff and order equal to 4. We compared in this case Model WF with a single RP
(the longest available) with respect to model WF using the 8 available reduced
proxies, where the comparison was made under the five dimension reduction
methods explained above.
\begin{table}[h!]
  \centering
  {\small
\begin{tabular}{lll|rrr|r}
  \toprule
  \textbf{Model} & $N$ & \textbf{Method} & IS$_{80}$ & IS$_{95}$ & CRPS & MSE \\ 
  \midrule
  WF & 1 & PCR & 0.4678 & 0.1792 & 0.1641 & 0.1476 \\ 
  WF & 1 & sPCR & 0.5821 & 0.2146 & 0.2593 & 0.2134 \\ 
  WF & 1 & LASSO & 0.4658 & 0.1780 & 0.1655 & 0.1744 \\ 
  WF & 1 & sPLS & 0.4749 & 0.1813 & 0.1810 & 0.1647 \\ 
  WF & 1 & SIR & 0.4779 & 0.1819 & 0.1671 & 0.1692 \\
  \midrule
  WF & 8 & PCR & 0.2493 & 0.0840 & 0.0986 & 0.0392 \\ 
  WF & 8 & sPCR & 0.1969 & 0.0719 & 0.0745 & 0.0926 \\ 
  WF & 8 & LASSO & 0.5044 & 0.3128 & 0.1559 & 0.1735 \\ 
  WF & 8 & sPLS & 0.3436 & 0.1687 & 0.1177 & 0.1579 \\ 
  WF & 8 & SIR & 0.6485 & 0.4797 & 0.1871 & 0.2124 \\
  \midrule
  NF & 8 & PCR & 0.2966 & 0.0958 & 0.1249 & 0.0275 \\ 
  NF & 8 & sPCR & 0.4275 & 0.1624 & 0.1600 & 0.0249 \\ 
  NF & 8 & LASSO & 0.5434 & 0.3545 & 0.1644 & 0.1828 \\ 
  NF & 8 & sPLS & 0.2617 & 0.0942 & 0.0967 & 0.1326 \\ 
  NF & 8 & SIR & 0.5772 & 0.3955 & 0.1720 & 0.1964 \\
  \midrule
  Mixed & 8 & PCR & 0.3509 & 0.1101 & 0.1579 & 0.0240 \\ 
  Mixed & 8 & sPCR & 0.3660 & 0.1368 & 0.1357 & 0.0249 \\ 
  Mixed & 8 & LASSO & 0.5291 & 0.3385 & 0.1613 & 0.1791 \\ 
  Mixed & 8 & sPLS & 0.3131 & 0.1404 & 0.1101 & 0.1495 \\ 
  Mixed & 8 & SIR & 0.5986 & 0.4209 & 0.1766 & 0.2019 \\ 
   \bottomrule
\end{tabular}
}
\caption{Comparison of predictive measures. The IS and CRPS measures are
  negatively oriented (lower scores are better).}
\label{tab:comparisontot}
\end{table}


% \begin{table}
% \centering
% \begin{tabular}{lll|rrr|rrr}
%   \toprule
%     \multicolumn{3}{c}{} & \multicolumn{3}{|c}{Observed Anomalies}  & \multicolumn{3}{|c}{PS04 dataset} \\
%   \midrule
%   \textbf{Model} & $N$ & \textbf{Method} & IS$_{80}$ & IS$_{95}$ & CRPS & IS$_{80}$ & IS$_{95}$ & CRPS  \\ 
%   \midrule
%   WF & 1 & PCR & 0.4678 & 0.1792 & 0.1641 & 0.5422 & 0.1792 & 0.2770 \\ 
%   WF & 1 & sPCR & 0.5821 & 0.2146 & 0.2593 & 0.6595 & 0.2147 & 0.3299 \\ 
%   WF & 1 & LASSO & 0.4658 & 0.1780 & 0.1655 & 0.5462 & 0.1780 & 0.2913 \\ 
%   WF & 1 & SPLS & 0.4749 & 0.1813 & 0.1810 & 0.5504 & 0.1812 & 0.2887 \\ 
%   WF & 1 & SIR & 0.4779 & 0.1819 & 0.1671 & 0.5509 & 0.1818 & 0.2881 \\
%   \midrule
%   WF & 8 & PCR & 0.2493 & 0.0840 & 0.0986 & 0.3802 & 0.1134 & 0.1617 \\ 
%   WF & 8 & sPCR & 0.1969 & 0.0719 & 0.0745 & 0.6900 & 0.2468 & 0.2455 \\ 
%   WF & 8 & LASSO & 0.5044 & 0.3128 & 0.1559 & 1.5297 & 1.2913 & 0.4153 \\ 
%   WF & 8 & SPLS & 0.3436 & 0.1687 & 0.1177 & 1.3822 & 1.1036 & 0.3847 \\ 
%   WF & 8 & SIR & 0.6485 & 0.4797 & 0.1871 & 1.7658 & 1.5793 & 0.4671 \\
%   \midrule
%   NF & 8 & PCR & 0.2966 & 0.0958 & 0.1249 & 0.3224 & 0.1026 & 0.1399 \\ 
%   NF & 8 & sPCR & 0.4275 & 0.1624 & 0.1600 & 0.4272 & 0.1624 & 0.1505 \\ 
%   NF & 8 & LASSO & 0.5434 & 0.3545 & 0.1644 & 1.5906 & 1.3666 & 0.4285 \\ 
%   NF & 8 & SPLS & 0.2617 & 0.0942 & 0.0967 & 1.1900 & 0.8772 & 0.3433 \\ 
%   NF & 8 & SIR & 0.5772 & 0.3955 & 0.1720 & 1.6679 & 1.4583 & 0.4459 \\
%   \midrule
%   Mixed & 8 & PCR & 0.3509 & 0.1101 & 0.1579 & 0.3233 & 0.1120 & 0.1338 \\ 
%   Mixed & 8 & sPCR & 0.3660 & 0.1368 & 0.1357 & 0.3634 & 0.1368 & 0.1375 \\ 
%   Mixed & 8 & LASSO & 0.5291 & 0.3385 & 0.1613 & 1.5672 & 1.3377 & 0.4234 \\ 
%   Mixed & 8 & SPLS & 0.3131 & 0.1404 & 0.1101 & 1.3198 & 1.0284 & 0.3713 \\ 
%   Mixed & 8 & SIR & 0.5986 & 0.4209 & 0.1766 & 1.7001 & 1.4973 & 0.4530 \\ 
%    \bottomrule
% \end{tabular}
%   \caption{Comparison of predictive measures.}
%   \label{tab:comparisontot}
% \end{table}


An improvement is evident in all measures when we use all the available reduced proxies over the case
$N=1$, under the methods SPLS, PCR and sPCR for the first validation period.
For the PS04 dataset validation, all the methods except SIR show an improvement in terms of MSE. (Table~\ref{tab:comparisontot}, last column). Also note that, among the models with external forcings, the best performances from the viewpoint of these prediction measures are obtained with sPCR and PCR.

\subsection{Impact of Model Choice}
We are also interested in assessing whether a linear combination of B-spline bases can model GMST without the inclusion of external
forcings at all. It is clear from equations \eqref{eq:M1} and \eqref{eq:M3} that
one of the drawbacks of Models NF and Mixed is the arbitrariness of $K(\tau)$. We analyze the
relationship between the temperature observed during the calibration period
(1900-2000) and a linear combination of BSplines. The number of 
bases in Model NF is selected according to the adjusted $R^2$ of a linear regression model between observed anomalies and
the corresponding basis functions. Based on the above, we selected 6
BSpline functions for this period and we take $K(\tau)=120$ based on the
assumption that the number of BSpline bases is uniform
throughout the entire reconstruction period. Note that the high prevalence of
annually-resolved proxy data allows to assume that a constant number of BSpline bases might be adequate to describe the temperature
mean.%\jeg{this assumption seems strange given that the number of proxies varies drastically. Can you justify it?}\lb{I tried using the fact that practically  all the data here comes in an annual basis, so the deterministic trend of the temperature can be modelled in an uniform way}.\jeg{I supposed the RPs are annual, even if they are made of fewer and fewer observations as you go further back in time. I'll be curious to see what the reviewers say about this, but I don't want this to hold up the show.} \jeg{I am still not sure I understand, but I'll let it go}
  
  The choice of $K(\tau)=100$ for model Mixed is
based on the same criteria as before. Finally, we fit the models NF and Mixed
with the previous choices of $K(\tau)$, using the five different data reduction
techniques described in section \ref{sec:rp}. These results are
also shown in Table \ref{tab:comparisontot}. Note that the SPLS method achieves a better performance in terms of the
predictive measures for the first validation period, and PCR gets the smallest
MSE for the second period. 

 Four reconstructions of Common Era GMST
are shown in Figure \ref{fig:paleoCE1}.
In order to illustrate the reconstruction that we obtained for both models WF and Mixed, we considered the best two choices in terms of the validation measures for the
first out-of-sample period  and the best two choices for the second testing
period: sPCR and PCR methods for both the WF and Mixed models.

\begin{figure}[h!]
  \centering
  \includegraphics[scale=0.40]{RecCE_Final}
  \caption{Paleoclimate Reconstruction in the Common Era (CE) with 95\%
    prediction bands. Best two choices per validation method.}
  \label{fig:paleoCE1}
\end{figure}

%\jeg{I take issue with the definition of "best" here. This is entirely contingent on the choice of calibration and verification period, and many studies have shown that you can obtain different results with different defintions of such periods. This is basically a cross-validation exercise, and there would be many different choices (5-fold cross validation is my personal favorite). I would strongly urge against declaring any ``best'' choices based  only on the short observational record. Inevitably we need to assert our favorites, but let us not fool ourselves that this is an objective  choice.}\lb{I think you're completely right, let's focus this idea in the discussion section, because we have data constraints that won't let us choose the most adequate model.}

The best model for the first testing period (sPCR-WF) shows an interesting
balance in terms of the variance of the reconstructed series and the width of
the confidence region approximated by INLA. The remaining 3 reconstructions show
similar small-scale tendencies with respect to the best choice, but the width of
their confidence regions is greater. A closer look of the reconstructions is shown in Figure
\ref{fig:paleo19001}, Panel (A), where the first testing period appears between the red lines.
\begin{figure}[h!]
  \centering
  \subfloat[]{\includegraphics[scale=0.33]{Rec1900_Final}}
  \subfloat[]{\includegraphics[scale=0.33]{Rec1500_Final}}
  \caption{Panel (A): Paleoclimate Reconstruction 1850-2000 with 95\%
    prediction bands. The out-of-sample validation period is
    located between the red lines. Panel (B): Smoothed Reconstruction 1600-1900 vs PS04 series. Best two choices per validation method.}
  \label{fig:paleo19001}
\end{figure}
Note that sPCR-WF closely predicts the anomalies observed in the first testing
period and it does so with a higher level of accuracy than the other
reconstructions. In addition, the other three reconstructions underestimate in
an almost analogous way the anomalies observed during the same period. At least
for this validation exercise there is no clear advantage to using a B-Spline
basis as an additional linear term in equation \eqref{eq:M3}.

Figure \ref{fig:paleo19001}, Panel (B) shows a comparison of the smoothed reconstructions
using a low-pass filter and the borehole-based reconstruction of
\cite{Pollack2004} (PS04 dataset). In this case, the mixed versions of the sPCR
and PCR models with external forcings provide the best adjustment in terms of
the MSE measure, especially during the 1700-1850 period. We can infer therefore
that the B-Spline basis provides enough flexibility to attain reconstructed
anomalies with low-frequency features. Therefore, PCR-Mixed yields the most balanced
results in terms of these two validation exercises.
% Note that although the reconstruction we obtained in
% SPLS-SSnF has a fairly narrow confidence region, there are quite a few observed
% anomalies during the out-of-sample testing period (1850-1899) that are outside
% this range within the calibration period as well as the beginning of the 20th
% century. In addition, this reconstruction has smaller variance that the observed
% anomalies during both validation periods, which is not so noticeable in the other
% three best reconstructions and  it also overestimates the the observed series
% during the out-of-sample validation period. On the other hand, the reconstructions PCR-WF and PCR-SSnF  
% underestimate the observed anomalies considerably during the period 1850-1900,
% although the observed series during this period is completely contained in the
% respective confidence bands. For the above reasons we believe that the
% reconstruction that has the best results is the one obtained by sPCR-WF, followed
% by the PCR-WF model. 

% \begin{figure}[h!]
%   \centering
%   \includegraphics[scale=0.40]{Rec1500_Final}
%   \caption{Smoothed Reconstruction 1600-1900 vs PS04 series. Best two choices per validation method.}
%   \label{fig:paleo15001}
% \end{figure}

\subsection{Impact of INLA sampling}

\begin{figure}[h!]
  \centering
  \includegraphics[scale=0.40]{RecCE_MCMC}
  \caption{Paleoclimate Reconstruction in the Common Era (CE) with 95\%
    prediction bands. Model sPCR-WF under two methods: INLA and MCMC.}
  \label{fig:paleoCE4}
\end{figure}

We now quantify the trade-offs of approximating the MCMC procedure using INLA. The WF model in its simplest case (1 nest) was fitted in B14
using an MCMC approach. We employed this approach in order to adjust the
WF model with the first reduced proxy from the sPCR method (chosen purely
for comparison purposes). The MCMC was performed using the
same priors as in B14, but with a larger calibration period
(1900-2000). The results are shown in Figures~\ref{fig:paleoCE4} and
\ref{fig:betas}. Note that the MCMC reconstruction reaches cooler temperatures
than INLA's and that its confidence bands along the reconstruction period are
narrower. This last fact coincides with a small difference in terms of the
interval score measure. Despite these contrasts, the general trends of both reconstructions are quite similar. 

Another point of comparison is the estimated coefficients of the
external forcings in equation \eqref{eq:M2}. The estimated density function
for each coefficient is shown in Figure \ref{fig:betas}. The behavior of the
estimated parameters of the three external forcings is very similar between the
two methods: by far the most influential external forcing in
both reconstructions is the greenhouse gas component, followed by explosive volcanism. Consistent with \citet{Schurer2013b}, solar irradiance comes last, and is indistinguishable from a zero effect. 

For the same reason as above, the estimated density
function of the CO$_2$ coefficients are more
concentrated when we use an MCMC algorithm, to a lesser extent for the other 
coefficients. This behavior confirms the observed details of Figure
\ref{fig:paleoCE4}. 

\begin{figure}
  \centering
  \includegraphics[scale=0.3]{PlotBetas}
  \caption{Posterior densities of $\beta_1$, $\beta_2$ and $\beta_3$ for Model
    sPCR-WF and MCMC using PCR single reduced proxy.}
  \label{fig:betas}
\end{figure}

In terms of computational efficiency, the INLA procedure is quite remarkable, not only
for our case but in most of the previous work on a similar topic as well (see \cite{Rue2009},
\cite{Blangiardo2013}, \cite{Ruiz-Cardenas2012} for a few examples). The
computational cost of MCMC sampling with 5000 samples was
approximately 8 hours, whereas the computational time of INLA's best model with
a single reduced proxy was approximately 15 seconds. This comparison
was performed on an Ubuntu 16.04 server with Intel Xeon E5-2630 (8-cores,
2.40GHz) and 64 GB of RAM.  This large speedup allowed us to explore a far wider variety of modeling choices than MCMC alone.


\subsection{Are we living in extraordinary times?}

% \jeg{can we use the Mixed model as baseline here?. You could put the figures for NF and WF in the Appendix, if need be.}\lb{Because of Figures 6 and 7 we can use PCR-Mixed as the baseline in this section.} \jeg{thanks!}

    Finally, we take advantage of our probabilistic output to ask whether the levels and rates of warming over the recent past are exceptional in a 2,000 year context.  Fig.~\ref{fig:compmeans} compares the temporal means of the last 10, 25, 50 and 100 years of the record compared to all similar previous intervals. That is, for the 25y case, we compare the mean over 1976-2000 to the mean over 1951-1975, 1926-1950, and all other 25-year segments back to 1 CE. It is evident that in all four cases the most recent period appears detectably warmer than any other time in the past 2,000 years, which confirms and strengthens the original conclusions of \cite{MBH99}.  

\begin{figure}
  \centering
  \includegraphics[scale=0.35]{compMeans}
  \caption{Comparison of the distribution of reconstructed anomalies for different time horizons. Model PCR-Mixed. }
  \label{fig:compmeans}
\end{figure}

The situation is more nuanced for warming rates: there is no discernible difference on 10 or 20y intervals, but warming rates over 1951-2000 and 1901-2000 appear highly anomalous in a 2,000 year context. This is not surprising, as transient heat exchanges between the surface and deep ocean are known to occur on scales shorter than a few decades \citep{Hansen:2005}. These transients average out over 50-100y scales, and reveal a highly anomalous warming rate, which can be unequivocally tied to the radiative forcing due to anthropogenic emissions of $CO_2$ (parameter $\beta_3$ in the WF and Mixed models).

\begin{figure}
  \centering
  \includegraphics[scale=0.35]{compTrends}
  \caption{Comparison of the distribution of trends of reconstructed anomalies for different time horizons. Model PCR-Mixed.}
  \label{fig:comptrends}
\end{figure}

\section{Conclusions}
We carried out a Bayesian inference of global mean surface temperature over
the past 2,000 years. By leveraging INLA to lighten the computational burden, our framework allows us to investigate a wide range of model choices and data reduction strategies.  We validated the result using instrumental data over the 1850-1899 period, and using independent borehole temperature inversions over 1600-1899. The former validates the reconstruction precisely to an interannual scale, while the latter constrains centennial trends.  

Below are a few take-home messages:\\
     (a) The data reduction techniques provide roughly equivalent results, with
      sPCR and PCR performing marginally better than other methods.
    
     \noindent (b) The model choice is highly consequential. Model "Mixed" is the most
      physically justifiable, and it guarantees a balance between the
      validation measures on the first and second testing periods. This models also
      appears to perform at least as well as other choices. The 
      additional nonparametric terms in the mean component of equation \eqref{eq:M3} allows
      to capture long-memory behavior that is not included in the external forcings, which compensates for the independent structure of the errors. 
      %besides giving an additional dependence structure to the error term.
      
     \noindent (c) In cases where both INLA and MCMC are implemented, INLA allows to approximate the MCMC
      solution at a fraction of the computational cost, but with wider
      prediction intervals. This exercise was performed only for the simplest
      model available (1 reduced proxy), because of the prohibitive cost of the
      MCMC approach for more complex models like the ones presented in Figure
      \ref{fig:paleoCE1}. 
      
     \noindent (d) Adding to a wide body of literature, we find that current temperature levels are unprecedented in the past 2,000 years. 
      The twentieth century also stands out as warming more rapidly than any other
      century in the past 2,000 years. All models that include forcings (WF, Mixed) show that the man-made increase in atmospheric carbon dioxide is the leading contributor to this warming effect. 
%\end{itemize}

One limitation of our analysis is the strong
dependence of our results on the choice of the testing period. Ideally, a
cross-validation exercise should be carried out to determine with greater
certainty the expected prediction error of the models, but due to the restricted access to
additional comparison information the cross-validation is very challenging for this problem. 
Another way to improve our current study is to consider the space-time variability of the reduced proxies and temperatures in the model. 
Since INLA has been proven to be computationally
efficient, it could be used to extend this work to a spatiotemporal context.



%The impact of data reduction choices was explored by five popular dimension reduction techniques that allow us to compute reduced proxies under different temporal horizons and we were able to compare them through their predictive capacity during a validation period. We also considered the model developed in B14, and we explored to make it more complex by including more information from the proxy dataset, as well as by extending the reconstruction period by 1000 years. The participation of external forces in the hierarchical Bayesian model was tested by comparing its ability to describe the average behavior of temperature anomalies with respect to a linear combination of BSplines. Of course, this adds more complexity to the hierarchical model and  it becomes more necessary to solve the slow behavior of the adjustment process when we employ MCMC techniques. This second challenge was successfully solved by approximating the adjustment with INLA, with which we were able to adjust Bayesian models with much greater complexity than B14 in a much shorter period of time.  Discuss about using INLA for space-time reconstruction...



% The same can be said
% for a closer look of the reconstructions during the period 1900-2000 (see
% . For comparison purposes,
% we also include the reconstructions of Model WF with the PCR method (see the
% upper panels of 
% figures ), where the
% main difference corresponds to the period 1-250 in terms of the variance of the
% mean function, mainly due to the sensitivity of the B-Spline basis to
% discrepancies among reduced proxies in a fixed time period. Also note that the
% resulting posterior densities of the external forcings (see Figure
% \ref{fig:betas} in the Appendix) suggest that both $S_t$
% and $\tilde C_t$ are quite significant to explain the temperature anomalies.    




% Finally, we also compare the nature of the linear function that determines the
% mean behavior of the state equation in both models. As the coefficients that
% accompany the covariates are random (External forcings for Model WF, B-Splines
% for Model SSnF) then it is possible to calculate confidence bands for the mean
% function in both cases. This is illustrated in Figure \ref{fig:meanfunction}.
% \begin{figure}
%   \centering
%   \includegraphics[scale=0.4]{MeanFunction_comp}
%   \caption{Mean function of the state equation and its 95\% confidence band.}
%   \label{fig:meanfunction}
% \end{figure}
% Note that the variance and the stability of the mean function are much greater
% in the best case of Model SSnF than in the best case of Model WF. This could be an
% indicator that Model WF has a greater bias in its mean behavior, which is a
% negative aspect in the reconstruction as a whole.



%\printbibliography
\footnotesize
\bibliographystyle{apalike}
\bibliography{biblioteca}
%\documentclass[11pt]{amsart}
\usepackage{amsmath,amsfonts,amssymb,amsthm,verbatim,multirow,url,subfig,footnote,graphicx,array,xr,booktabs,placeins,float}
\usepackage[usenames,dvipsnames]{color}
%  \usepackage[style=numeric,
%  doi=false,
%  isbn=false,
%  url=false]{biblatex}
\usepackage[utf8]{inputenc}
\usepackage[T1]{fontenc}


\theoremstyle{plain}
\newtheorem{lema}{Lemma}
\newtheorem{coro}{Corollary}
\newtheorem{teo}{Theorem}
\newtheorem{prop}{Proposition}
\theoremstyle{definition}
\newtheorem{ap}{Assumption}
\newtheorem{defin}{Definition}
\theoremstyle{remark}
\newtheorem*{rk}{Remark}
\newcommand{\nn}{\mathbf}
\newcommand{\nns}{\boldsymbol}
\newcommand{\Hcal}{\mathcal H}

\newcommand{\fv}[1]{\color{ForestGreen}\textbf{[FV: #1]}\normalcolor}
\newcommand{\lb}[1]{\color{MidnightBlue}\textbf{[LB: #1]}\normalcolor}

%\addbibresource{biblioteca.bib}

\begin{document}
\title[Paleoclimate Reconstruction using INLA.]{Efficient Reconstructions of Common Era Climate via Integrated Nested Laplace Approximations.}

\author{Luis A. Barboza}
\address{Centro de Investigacion en Matematica Pura y Aplicada (CIMPA)-Escuela
  de Matematica, Universidad de Costa Rica\\
San Jos\'e, Costa Rica}
\email{luisalberto.barboza@ucr.ac.cr}


\author{Julien Emile-Geay}
\address{Department of Earth Sciences \\
  University of Southern California \\
  Los Angeles, California, USA.
}
\email{julieneg@usc.edu}

\author{Wan He}

\author{Bo Li}
\address{Department of Statistics \\
  University of Illinois at Urbana-Champaign \\
  Champaign, Illinois, USA.
}
\email{libo@illinois.edu}



\date{\today}
%\date{December 20, 2014}
\keywords{INLA,Paleoclimate Reconstruction,Hierarchical Bayesian Model}
\subjclass[2010]{}
\maketitle

\begin{abstract}
A Paleolimate Reconstruction on the Common Era (1-2000AD) was performed using a
Hierarchical Bayesian Model from three sources of data: proxy data from PAGES2k
project dataset, HadCRUT4 temperature data from the Climatic Research Unit
at the University of East Anglia and external forcing data from several sources.
Instead of using the MCMC approach to solve for the latent variable, we used the
INLA algorithm that shows a similar approximation than previous studies. The use
of external forcings was tested by replace them with a fixed number of
BSplines in the latent equation. Classical goodness-of-fit measures show that there is not a significant
difference between the predictive ability of both approaches. 
\end{abstract}

\section{Introduction.}
\label{sec:intro}

\begin{itemize}
\item PAGES2k dataset.
  
\item Reduction methods.
  
\item External forcings.
  
\item INLA efficiency.
\end{itemize}


%\section{Integrated Nested Laplace Approximation (INLA).}
%\label{sec:inla}

The INLA approach uses a general specification where the mean of a sequence of
observations is a function of the linear structure:
\begin{align}\label{eq:meanINLA}
  \eta_i = \alpha +\sum_{m-=1}^M\beta_mx_{mi}+\sum_{l=1}^Lf_l(z_{li})
\end{align}
where $\alpha$ represents an intercept; the coefficients
$\mathbf{\beta} = (\beta_1,\ldots,\beta_M)$ are related to $M$ covariates
$(x_1,\ldots,x_M)$ and $f = \{f_1(\cdot),\ldots,f_L(\cdot)\}$ is a collection of
random effects defined on a set of $L$ covariates $(z_1,\ldots,z_L)$ (see
\cite{Rue2009} and \cite{Blangiardo2013}). Denote the set of random parameters as
$\theta = (\alpha,\beta,f)$ with $K$ hyperparameters $\psi =
\{\psi_1,\ldots,\psi_K\}$. If $y=(y_1,\ldots,y_n)$ denotes the observations, we
can assume conditional independence in the following way:
\begin{align*}
  p(y|\theta,\psi)=\prod_{i=1}^np(y_i|\theta_i,\psi),
\end{align*}
The INLA approach assumes that (1) the vector $\theta$ .is multivariate normal with
some precision matrix depending on the hyperparameters $\psi$, and (2) this vector
$\theta$ is conditionally independent given the hyperparameters. These two
assumptions specifies $\theta$ as a Gaussian Markov random field.

The main objectives of the bayesian estimation in this case are to compute the
marginal posterior distribution of each parameter in $\theta$:
\begin{align*}
  p(\theta_i|y) = \int p(\theta_i,\psi|y)d \psi = \int p(\theta_i|\psi,y)p(\psi|y)d \psi
\end{align*}
and the marginal posterior distribution of each hyperparameter:
\begin{align*}
  p(\psi_k|y)=\int p(\psi|y) d\psi_{-k}.
\end{align*}
In order to compute the above, we need to approximate the components $p(\psi|y)$
and $p(\theta_i|\psi,y)$. The first component can be approximated using a
Laplace Approximation (see \cite{Tierney1986}):
\begin{align*}
  p(\psi|y)&=\frac{p(\theta,\psi|y)}{p(\theta|\psi,y)}\propto \frac{p(\psi)p(\theta|\psi)p(y|\theta)}{p(\theta|\psi,y)}\\
           & \approx  \frac{p(\psi)p(\theta|\psi)p(y|\theta)}{\tilde p(\theta|\psi,y)} \Biggm |_{\theta=\theta^*(\psi)} := \tilde p(\psi|y)
\end{align*}
where $\tilde p(\theta|\psi,y)$ is the Gaussian approximation of
$p(\theta|\psi,y)$ and $\theta^*(\psi)$ is its mode (see \cite{Rue2009}). The
second component can be approximated in a similar way:
\begin{align}\label{eq:second}
  p(\theta_i|\psi,y)&=\frac{p((\theta_i,\theta_{-i})|\psi,y)}{p(\theta_{-i}|\theta_i,\psi,y)} \notag \\
  &\approx \frac{p((\theta_i,\theta_{-i})|\psi,y)}{\tilde p(\theta_{-i}|\theta_i,\psi,y)} \Biggm |_{\theta_{-i}=\theta_{-i}^*(\theta_i,\psi)}:=\tilde p(\theta_i|\psi,y)
\end{align}
and $\tilde p(\theta_{-i}|\theta_i,\psi,y)$ is the gaussian approximation of
$p(\theta_{-i}|\theta_i,\psi,y)$. In this case $\theta=(\theta_i,\theta_{-i})$.
This last approximation has good precision, but it is quite complex because it
requires to recompute the above term for each value of $\theta$ and $\psi$. A
more efficient approach is the use the simplified Laplace Approximation which is
based on a Taylor's expansion of $\tilde p(\theta_i|\psi,y)$ in equation
\eqref{eq:second}. As mentioned in \cite{Rue2009} and \cite{Blangiardo2013},
INLA first explores the marginal joint posterior for the hyperparameters $\tilde
p(\psi | y)$ in order to locate the mode and then a grid search is performed to
produce a set of ``relevant'' points $\{\psi^*\}$ together with a set of weights
$w_{\psi^*}$ as an approximation of this marginal distribution. The marginals
$p(\psi^*|y)$ are refined by using interpolation methods. Finally the marginals
$\tilde p(\theta_i|y)$ are obtained as follows:
\begin{align*}
  \tilde p(\theta_i|y) \approx \sum_{\psi^*}\tilde p(\theta_i|\psi^*,y)\tilde p(\psi^*|y)w_{\psi^*}.
\end{align*}

\section{Datasets.}
\label{sec:data}

\subsection{Proxy data.}
The PAGES2k global multiproxy database is a ``community-driven effort to
synthesize all publicly-archived, temperature-sensitive proxy records of the
past 2,000 years'' (see \cite{Kaufman2014}, \cite{PAGES2KConsortium2013} and \cite{PAGES2kConsortium2017}). The main objective of this database is
to create a free information conglomerate that integrates temperature proxies
with different level of resolution, in order to develop climatic
reconstructions.

The dataset is composed by a collection of borehole, coral, documentary, glacier
ice, lake and marine 
sediment, sclerosponge, speleothem and tree-ring data; collected through 688
data series around the world. Each of those proxies
has different time horizons and this creates difficulties in the aggregation of
information in simpler variables.  In order to select proxies with high
predictive power, we first chose those series with large correlation with respect to
their closest spatial temperature record using the HadCRUT4.2 dataset. More
details on this ``screening'' procedure can be found in \cite{Emile-Geay2015}. Finally we get 257 proxies
that pass the previous procedure (see the first panel of Figure \ref{fig:proxy} with the spatial
distribution of the final proxies).   
\begin{figure}
  \centering
  \includegraphics[scale=0.45]{CombinedMap_Area}
  \caption{Pages2k proxy distribution and temporal availability of proxies by
    type, after the screening procedure of \cite{Emile-Geay2015}.}
  \label{fig:proxy}
\end{figure}
As mentioned, proxies have different temporal horizons, as it can be shown in the
second panel of Figure~\ref{fig:proxy}. Unlike previous studies (for example \cite{Barboza2014}), in this article we try to take into account the information available in most proxies, despite their temporal diversity.


\subsection{Temperature data.}
We used the HadCRUT4 global temperature dataset provided by the Met Office Hadley
Centre and the Climatic Research Unit at the University of East Anglia, UK (version 4.4.0.0). The
data consists of historical information of temperature anomalies relative to the
period 1961-1990 in degrees Celsius, on an annual basis and calculated as
medians of spatial information (more details in \cite{Morice2012}).

\subsection{Forcing data.}
The forcing data consists of (see Figure \ref{fig:forcings} in the Appendix):
\begin{itemize}
\item Historical Greenhouse-Gases concentrations: hemispheric means of mole
  fraction of carbon dioxide in air (ppm) with annual
  resolution, taken from Coupled Model Intercomparison Project (CMIP6) (see
  \cite{Meinshausen2016}).
  
\item Volcanic forcing from Easy Volcanic Aerosol (EVA) dataset (evolv2k): (see
  \cite{Toohey2016}) reconstructed zonal mean AOD (mid-visible, i.e., 550 nm), covering the
  500 BCE to 1900 CE time period. For 1900 (or 1850) to present, \cite{Thomason2016} is
  used to fill in the forcing table, the volcanic data at varying locations is
  weighted by cosine of their corresponding latitudes.
  
\item The solar forcing data is computed from SATIRE-H
  (Holocene) at \cite{Vieira2011}. Irradiance from SATIRE-H (Holocene) is
  recorded on a decadal basis from 9495BC - 1939AD and then on a daily basis
  from 1940AD onwards. Hence, the data prior to 1939AD was interpolated using splines and after 1940AD, an annual mean is computed. For consistence of resolution of the data, annual data from 1640AD to 2000AD is also spline interpolated.
\end{itemize}
\textbf{Julien: Can you please include more details on this section?}
\section{Hierarchical Bayesian Model.}
\label{sec:model}

\subsection{Reduced Proxy.}
\label{sec:rp}
Following \cite{Barboza2014}, we reduce the dimension of the proxy dataset
through the notion of a Reduced Proxy ($RP$). First of all, we centered and
scaled the 257 proxy variables. Then we remove proxy series whose proportion of lost
annual 
observations is larger than 5\%. Due to the diversity of start
dates in the proxies database (see Figure \ref{fig:proxy}), we divide
proxies into non-homogeneous groups where each group has temporal availability within an
interval of 250 years. As the reconstruction is taking place over a 2000 year
horizon, this creates 8 groups with the distribution shown in Table \ref{tab:distdate}.
\begin{table}
  \centering
  \begin{tabular}{c|c|c}
    \toprule
    Group & Interval (year AD) & Number \\
    \midrule
    1 & 1-250 & 19 \\
    2 & 251-500 & 25 \\
    3 & 501-750 & 29 \\
    4 & 751-1000 & 33 \\
    5 & 1001-1250 & 54 \\
    6 & 1251-1500 & 65 \\
    7 & 1501-1750 & 105 \\
    8 & 1751-2000 & 146 \\
    \bottomrule
  \end{tabular}
  \caption{Distribution of proxies according to their temporal availability.}
  \label{tab:distdate}
\end{table}
Two important aspects that we can notice from the distribution of Table
\ref{tab:distdate} are: (1) in the last two intervals the number of proxies is greater than
the number of available observations in the calibration period defined as
1900-2000CE \footnote{The choice of the calibration period is based on the
  arguments shown in \cite{Barboza2014}} (101 observations) and (2) the number of proxies is very close to the
number of available observations, which can cause overfitting issues or
dimensionality problems in the use of classical linear models. Due to the above
reasons we adjusted five reduction techniques with the aim of adjusting
a linear model between the observed anomalies and the corresponding proxies in
Table \ref{tab:distdate} as covariates, all during the calibration
period. We used the following methods mainly because of their popularity:
\begin{description}
\item[Lasso Regression (LR)]
We used a 10-fold Lasso regression where the smoothing parameter was selected
through a cross validation criteria (see \cite{Tibshirani1996} and \cite{Friedman2010} for more details). % In this case, the Lasso
% regression was fitted using the temperature anomalies observed within the
% calibration period as the dependent
% variable and the corresponding standardized proxies in each of the groups
% defined in Table \ref{tab:distdate} as covariates.
 The Lasso regression penalizes the usual sum of squares with an argument
 containing the sum of the absolute values of each coefficient in the classical
 linear regression model, multiplied by an additional parameter (see \cite{Tibshirani1996}). Due
 to the geometric nature of the term of penalization, the search of estimators
 tends to assign values very close to zero to variables that have almost null
 effects with respect to the dependent variable. Clearly, this is its main advantage
 over other methods.
% In this case, the Lasso
% regression was fitted using the temperature anomalies observed within the
% calibration period as the dependent
% variable and the corresponding standardized proxies in each of the groups
% defined in Table \ref{tab:distdate} as covariates. 
\item[Sparse Partial Least Squares (sPLS)] 
  Partial least squares seeks to reduce the dimension of the design matrix in
  linear regression models with high-dimensionality issues through a latent matrix
  whose components maximize the linear correlation with respect to the
  dependent variable together with its variance. The sPLS method has the advantage to include sparseness in the partial least squares
  estimators, in order to avoid inconsistency problems when there are a
  considerable number of noisy covariates (see \cite{Chun2010} and \cite{Chung2013}). The
  thresholding parameter of this method is found using a 10-fold cross-validation criteria.    
\item[Sliced Inverse Regression (SIR) with CSS selection method]
  This procedure allows us to select reduced models which are similar to the
  complete model in terms of the linear correlation among SIR-QZ indices (see
  \cite{Coudret2014} and \cite{Coudret2017}). In general, the SIR methods (\cite{Li1991},
  \cite{Duan1991}, \cite{Zhong2005}, \cite{Li2008}, \cite{Coudret2014}, \cite{Weisberg2002} among
  others) reduce the dimension excess in non-parametric problems through the
  estimation of the linear space spanned by the coefficients of the covariables.
  We studied the association between proxies and temperatures
  through a linear regression among the chosen EDR directions using marginal
  dimension tests and the anomalies observed
  during the calibration period. 
\item[Principal Component Regression (PCR)]
We fitted a regression model between the temperature in the calibration period and the PCs
as covariates. We selected the number of principal components needed in each of
the eight regressions as the minimum number that attains, for the first time, an adjusted
$R^2$ of at least 70\% in each case. 
\item[Supervised Principal Components (sPCR)]
The PCR technique has been criticized in previous articles, see for example
\cite{Jolliffe1982} and \cite{Tibshirani1996} because the principal components
with a small contribution of variance are not necessarily the ones that
associate least with the dependent variable in a linear regression model.
Because of this, \cite{Bair2006} defines a technique where PCR is applied only
on a certain subset of covariates that have a considerable degree of association
with respect to the dependent variable, and this level is chosen through cross-validation.
\end{description}
Finally, we computed the reduced proxies by predicting the anomalies in the
reconstruction period under any of the above methods and any of the groups in
Table \ref{tab:distdate}. The eight reduced proxies for each methodology are
shown in the Figure \ref{fig:RPs} in the Appendix. Also note that
the obtained reduced proxies are highly correlated among methods, but the PCR
method shows a lower level of correlation compared with the rest of the methods,
as we can see in Figure \ref{fig:CorrRPs}.  

\subsection{Model Specification}
\label{sec:modelspec}

Our models are based on the one presented in \cite{Barboza2014}. Now we define
some notation, assuming that all the variables are set at time $t$.
\begin{itemize}
\item $RP_t^i$: $i$-th reduced proxy at time $t$.
  
\item $T_t$: temperature anomaly at time $t$.
  
\item $\tilde C_t = \log (C_t)$. Transformed greenhouse gases. The log
  transformation is chosen to approximate the radiative forcing due to changes
  in the equivalent CO$_2$ concentration. (see \cite{Barboza2014})
  
\item $\tilde V_t = \log (-V_t+1)$. Transformed volcanic forcing. More details
  on the choice of the transformation in \cite{Barboza2014}.
  
\item $B_t^{k,\tau}$. $k$-th B-Spline basis function at time $t$ with knot
  sequence $\tau$ (more details in \cite{DeBoor2001} and \cite{Ramsay2005}). We assume that the
  B-Spline basis is composed by cubic polynomials.  
\end{itemize}
With the above notation we can define two types of model as extensions of the
one defined in \cite{Barboza2014}, taking advantage of more reduced proxies:
\begin{description}
\item[State-Space model without forcings (Model SSnF)]
Basically, we would be defining a data level (according to the hierarchical bayesian models' jargon)
  for each available $RP$:
\begin{align}\label{eq:ssnf}
  \begin{cases}
    RP_t^i&=\alpha_0^i+\alpha_1^iT_t+\epsilon^i_t\\
  T_t&=\beta_0+\sum_{k=1}^{K(\tau)}\beta_k B_t^{k,\tau}+\eta_t
  \end{cases}
\end{align}
where $\{\alpha^i_j\}$ and $\{\beta_k\}$ are random parameters for
$i=1,\ldots,N$, $j=0,1$ and $k=1,\ldots,K(\tau)$. The number of reduced proxies is
$N\geq 1$. For simplicity, the error terms
$\epsilon^i_t$ and $\eta_t$ are assumed to be
independent normally-distributed random variables with finite variances
$\{\sigma^2_{\epsilon^i}\}$ and $\sigma^2_{\eta}$ respectively. These variances
are also considered as random and do not depend on time $t$. The main idea of
this model is to evaluate the central hypothesis of this article, that is, the
paleoclimate reconstruction can be performed without taking into account the
external forcings, as long as they are replaced with deterministic
functions that allow to describe the mean behavior of the series of anomalies.
\item[State-Space model with forcings (Model SSwF)]
As a second model we substitute the mean function in the second equation of
\eqref{eq:ssnf} with a linear combination of the external forcings:    
\begin{align}\label{eq:sswf}
  \begin{cases}
    RP_t^i&=\alpha_0^i+\alpha_1^iT_t+\epsilon^i_t\\
  T_t&=\beta_0+\beta_1S_t+\beta_2\tilde V_t+\beta_3\tilde C_t+\eta_t
  \end{cases}
\end{align}
\end{description}
The models \eqref{eq:ssnf} and \eqref{eq:sswf} are defined for
$t=1,\cdots,2000$ (Common Era). It is also important to add that we focus on
models with independent error structures since in \cite{Barboza2014} the authors
found that the greatest impact on the predictive capacity of these hierarchical
models is obtained when the forcings are added and not so much when the error
structures are more complex.

\section{Results.}
\label{sec:results}

The models proposed in equations \eqref{eq:ssnf} and \eqref{eq:sswf} were
implemented using the R package \textbf{r-inla}\footnote{www.r-inla.org}. The
implementation was based on the recommendations of \cite{Ruiz-Cardenas2012} and
\cite{Muff2015} on the use of the INLA methodology in state-space models,
dynamic linear models and, in general, models whose mean can be written
according to equation \eqref{eq:meanINLA}. These
recommendations together with \cite{Martins2013} allow
great flexibility when we work with bayesian models with several levels of
information, in particular in the case of levels of hierarchical models.

Like any hierarchical bayesian model, the most basic level of information
corresponds to the level of prior information. This is defined through the
following hypothesis:
\begin{itemize}
\item $\alpha^i_j\sim N(0,3)$, $\beta_\ell \sim N(0,3)$ for $i=1,\ldots,N$, $j=0,1$ and $\ell=0,\ldots,3$
  (Model SSwF) or $\ell=0,\ldots,K(\tau)$ (Model SSnF). The choice of the variance is
  completely arbitrary, but the main idea is to select a relatively large one.
  
\item $\rho_i := -\log \sigma^2_{\epsilon^i}\sim \text{log-gamma}(1,10^{-20})$
  (very small precision) for $i=1,\ldots,N$.
  
\item $\rho_0 := -\log \sigma^2_\eta \sim \text{log-gamma}(1,10^{-20})$ (very
  small precision).
\end{itemize}

As a first exercise, we analyze the change in the predictive capacity of the
reconstruction model when more equations involving proxies are included. We used
two scoring rules in \cite{Gneiting2007a} as measures of predictive ability:
IS$_\alpha$ (Interval Score at $\alpha$ level) and CRPS (Continuous Ranked
Probability Score). These scores have been previously employed in the
verification of point forecasts in environmental sciences for example, as well as the area
of paleoclimatic reconstructions (see \cite{Barboza2014} and
\cite{Scheuerer2014}). Table \ref{tab:comparisontot} contains the predictive
measures using the observed anomalies and INLA's prediction intervals over two
testing periods: 1900-2000 as an in-sample validation interval and 1850-1899 as an
out-of-sample validation period. We compared in this case Model SSwF with a single RP
(the longest available) with respect to model SSwF using the 8 available reduced
proxies, where the comparison was made under the five dimension reduction
methods explained above.

\begin{table}
  \centering
  \begin{tabular}{lll|rrr|rrr}
    \toprule
    \multicolumn{3}{c}{} & \multicolumn{3}{|c}{In-sample}  & \multicolumn{3}{|c}{Out-of-sample} \\
    \midrule
    \textbf{Model} & $N$ & \textbf{Method} & IS$_{80}$ & IS$_{95}$ & CRPS & IS$_{80}$ & IS$_{95}$ & CRPS \\
    \midrule
   SSwF & 1 & PCR & 0.4702 & 0.1799 & 0.1827 & 0.4678 & 0.1792 & 0.1641 \\ 
   SSwF & 1 & sPCR & 1.1464 & 0.3635 & 0.4700 & 0.5821 & 0.2146 & 0.2593 \\ 
   SSwF & 1 & LASSO & 0.4665 & 0.1786 & 0.1631 & 0.4658 & 0.1780 & 0.1655 \\ 
   SSwF & 1 & SPLS & 0.4750 & 0.1817 & 0.1844 & 0.4749 & 0.1813 & 0.1810 \\ 
   SSwF & 1 & SIR & 0.4763 & 0.1822 & 0.1736 & 0.4779 & 0.1819 & 0.1671 \\
   \midrule
   SSwF & 8 & PCR & 0.2166 & 0.0828 & 0.0727 & 0.2493 & 0.0840 & 0.0986 \\ 
   SSwF & 8 & sPCR & 0.1935 & 0.0719 & 0.0722 & 0.1969 & 0.0719 & 0.0745 \\ 
   SSwF & 8 & LASSO & 0.3382 & 0.1994 & 0.1103 & 0.5044 & 0.3128 & 0.1559 \\ 
   SSwF & 8 & SPLS & 0.2669 & 0.1294 & 0.0945 & 0.3436 & 0.1687 & 0.1177 \\ 
   SSwF & 8 & SIR & 0.4110 & 0.2845 & 0.1250 & 0.6485 & 0.4797 & 0.1871 \\
   \midrule 
   SSnF & 8 & PCR & 0.2496 & 0.0954 & 0.0820 & 0.2966 & 0.0958 & 0.1249 \\ 
   SSnF & 8 & sPCR & 0.6757 & 0.2082 & 0.2715 & 0.4275 & 0.1624 & 0.1600 \\ 
   SSnF & 8 & LASSO & 0.3569 & 0.2210 & 0.1143 & 0.5434 & 0.3545 & 0.1644 \\ 
   SSnF & 8 & SPLS & 0.2230 & 0.0938 & 0.0843 & 0.2617 & 0.0942 & 0.0967 \\ 
   SSnF & 8 & SIR & 0.3754 & 0.2426 & 0.1180 & 0.5772 & 0.3955 & 0.1720 \\
   \bottomrule
\end{tabular}
  \caption{Comparison of predictive measures.}
  \label{tab:comparisontot}
\end{table}


  % \begin{tabular}{lll|rrr}
  %   \toprule
  %   \textbf{Model} & $N$ & \textbf{Method} & IS$_{80}$ & IS$_{95}$ & CRPS \\
  %   \midrule
  %   SSwF & 1 & PCR & 0.4702 & 0.1799 & 0.1704 \\ 
  %   SSwF & 1 & LASSO & 0.4665 & 0.1786 & 0.1352 \\ 
  %   SSwF & 1 & SPLS & 0.4750 & 0.1817 & 0.1809 \\ 
  %   SSwF & 1 & SIR & 0.4763 & 0.1822 & 0.1986 \\
  %   \midrule
  %   SSwF & 8 & PCR & 0.2166 & 0.0828 & 0.0685 \\
  %   SSwF & 8 & LASSO & 0.3382 & 0.1994 & 0.1061 \\ 
  %   SSwF & 8 & SPLS & 0.2669 & 0.1294 & 0.0919 \\ 
  %   SSwF & 8 & SIR & 0.4110 & 0.2845 & 0.0959 \\
  %   \midrule
  %   SSnF & 8 & PCR & 0.1809 & 0.0606 & 0.0449 \\ 
  %   SSnF & 8 & LASSO & 0.2792 & 0.1368 & 0.1016 \\ 
  %   SSnF & 8 & SPLS & 0.2087 & 0.0822 & 0.0730 \\ 
  %   SSnF & 8 & SIR & 0.2641 & 0.1203 & 0.0960 \\
  %   \bottomrule
  % \end{tabular}
  
% \begin{tabular}{lll|rrr}
%   \toprule
%  \textbf{Model} & $N$ & \textbf{Method} & IS$_{80}$ & IS$_{95}$ & CRPS \\ 
%   \midrule
%   SSwF & 1 & PCR & 0.5249 & 0.1988 & 0.2278 \\
%   SSwF & 1 & LASSO &  &  &  \\
%   SSwF & 1 & SPLS &  &  &  \\
%   SSwF & 1 & SIR &  &  &  \\
%   \midrule
%   SSwF & 8 & PCR & 0.1622 & 0.0542 & 0.0702 \\ 
%   SSwF & 8 & LASSO & 0.2465 & 0.1180 & 0.0864 \\ 
%   SSwF & 8 & SPLS & 0.1755 & 0.0586 & 0.0687 \\ 
%   SSwF & 8 & SIR & 0.2439 & 0.1144 & 0.0891 \\
%   \midrule
%   SSnF & 8 & PCR & 0.1583 & 0.0532 & 0.0435 \\ 
%   SSnF & 8 & LASSO & 0.1881 & 0.0640 & 0.0697 \\ 
%   SSnF & 8 & SPLS & 0.1819 & 0.0611 & 0.0574 \\ 
%   SSnF & 8 & SIR & 0.1649 & 0.0550 & 0.0462 \\ 
%    \bottomrule
% \end{tabular}



% \begin{table}
%   \centering
%   \begin{tabular}{c|rrr}
%     \toprule
%     Number of RPs-Method& IS$_{80}$ & IS$_{95}$ & CRPS \\ 
%     \midrule
%     1 RP & 0.5249 & 0.1988 & 0.2278 \\ 
%     8 RPs-PCR & 0.1622 & 0.0542 & 0.0702 \\ 
%     8 RPs-LASSO & 0.2465 & 0.1180 & 0.0864 \\ 
%     8 RPs-SPLS & 0.1755 & 0.0586 & 0.0687 \\ 
%     8 RPs-SIR & 0.2439 & 0.1144 & 0.0891 \\ 
%     \bottomrule
% \end{tabular}
%   \caption{Comparison of predictive measures in Model 1.}
%   \label{tab:comparison1}
% \end{table}
It is evident that there is a substantial improvement
in all the measures when we use all the available reduced proxies under any
method within the 1900-2000 period. However, for the out-of-sample validation, only
the SPLS, PCR and sPCR show an improvement in terms of measures.  Also note that, among the models with external forcings, the techniques that
guarantees better results in terms of prediction are sPCR and PCR.

We are also interested in verifying whether a linear combination of B-splines
models the average behavior of the anomalies without the inclusion of external
forcings at all. It is clear from equation \eqref{eq:ssnf} that one of the
drawbacks of Model 2 is the arbitrariness of $K(\tau)$. We analyze the
relationship between the temperature observed during the calibration period
(1900-2000) and a linear combination of BSplines. The number of elements of the
base is selected according to the adjusted $R^2$ of a linear regression model between observed anomalies and
the corresponding basis elements. Based on the above, we finally selected 6
BSpline terms for this period and we take $K(\tau)=120$ based on the
assumption that the number of BSpline terms is uniform
throughout all the reconstruction period. Finally we fit the model in
\eqref{eq:ssnf} with the previous choice of $K(\tau)$ using the five techniques
in section \ref{sec:rp}. These results are
shown in the last five lines of Table \ref{tab:comparisontot}. Due to the comments of the above paragraph, we opt to
use all the reduced proxies in this case.
% Therefore, for each of
% the methods in section 4.1, the IS and CRPS measures were calculated for a fixed
% value of $K(\tau)$ over a grid defined in [1,50]. After this, the best $K(\tau)$
% for each method was selected using the IS and CRPS measures. These results are
% shown in the last four lines in Table \ref{tab:comparisontot}. Due to the comments of the above paragraph, we opt to
% use all the reduced proxies in this case.
% \begin{table}
%   \centering
%   \begin{tabular}{c|rrr}
%   \toprule
% Method  & IS$_{80}$ & IS$_{95}$ & CRPS \\ 
%   \midrule
% PCR & 0.1583 & 0.0532 & 0.0435 \\ 
%   LASSO & 0.1881 & 0.0640 & 0.0697 \\ 
%   SPLS & 0.1819 & 0.0611 & 0.0574 \\ 
%   SIR & 0.1649 & 0.0550 & 0.0462 \\ 
%    \bottomrule
% \end{tabular}
%   \caption{Comparison of predictive measures of Model 2.}
%   \label{tab:comparison2}
% \end{table}
Note that the SPLS method achieves a better performance in terms of all the
predictive measures for both validation periods, even though the PCR results
differ very little in its corresponding measures. 

In order to illustrate the reconstruction that we obtained for both models, we
considered the best four choices in terms of the validation measures in Table
\ref{tab:comparisontot}. sPCR and PCR methods for the SSwF model and SSPLS and
PCR for
the SSnF model are the best choices. First note that the four respective reconstructions of the
temperature anomalies in the Common Era are shown in
Figure \ref{fig:paleoCE1}.
\begin{figure}
  \centering
  \includegraphics[scale=0.55]{RecCE_Final}
  \caption{Paleoclimate Reconstruction in the Common Era (CE) with 95\%
    prediction bands. Best four choices.}
  \label{fig:paleoCE1}
\end{figure}
The best choice (sPCR-SSwF) shows an interesting
balance in terms of the variance of the reconstructed series and the width of
the confidence region approximated by INLA. Reconstructions PCR-SSwF and
PCR-SSnF show similar small-scale tendencies with respect to the best choice, but the width of their confidence regions
is greater, which is an indicator of less precision in the overall
reconstruction. These last resconstructions also have smaller amplitude than the one shown by the best
option, this clearly is an indicator that the absence or presence of the
forcings is not decisive at the time of performing the reconstruction. In
general terms these three reconstructions have a very similar average level
of anomalies, except that the PCR-SSnF method gives anomalies that are generally a
bit colder than the two best ones. A closer look of the best reconstructions is shown in Figure
\ref{fig:paleo19001}.
\begin{figure}
  \centering
  \includegraphics[scale=0.55]{Rec1900_Final}
  \caption{Paleoclimate Reconstruction 1850-2000 with 95\%
    prediction bands. Best four choices. The out-of-sample validation period is
    located between the red lines.}
  \label{fig:paleo19001}
\end{figure}
Note that although the reconstruction we obtained in
SPLS-SSnF has a fairly narrow confidence region, there are quite a few observed
anomalies during the out-of-sample testing period (1850-1899) that are outside
this range within the calibration period as well as the beginning of the 20th
century. In addition, this reconstruction has smaller variance that the observed
anomalies during both validation periods, which is not so noticeable in the other
three best reconstructions and  it also overestimates the the observed series
during the out-of-sample validation period. On the other hand, the reconstructions PCR-SSwF and PCR-SSnF  
underestimate the observed anomalies considerably during the period 1850-1900,
although the observed series during this period is completely contained in the
respective confidence bands. For the above reasons we believe that the
reconstruction that has the best results is the one obtained by sPCR-SSwF, followed
by the PCR-SSwF model. 

The SSwF model in its simplest case (1 proxy) was fitted in \cite{Barboza2014}
using an MCMC approach. We employed this approach in order to adjust the
SSwF model with the first reduced proxy from the sPCR method (the best choice
obtained from Table \ref{tab:comparisontot}). The MCMC was performed using the
same priors as in \cite{Barboza2014}, but with a larger calibration period
(1900-2000). The results are shown in Figures \ref{fig:paleoCE4} and
\ref{fig:paleo19004} in the appendix, together with the results of the best
model according to Table \ref{tab:comparisontot}. Note that the MCMC reconstruction is cooler than INLA's, due
mainly to the inclusion of more information from the remaining 7 reduced
proxies. The MCMC confidence bands along the reconstruction period are narrower,
this can be an indicator that the whole reconstruction obtained through a MCMC technique has larger
bias than the one obtained with INLA method. During the out-of-sample validation
period we can be observe that the reconstruction using the MCMC method
underestimates the observed anomaly considerably, while in the case of INLA that
does not happen at all. In addition, the validation measures IS$_{80}$ and IS$_{90}$
during the same period are 0.3 and 0.16 respectively for the MCMC case, and they
differ a lot with respect to the ones obtained for the best model in Table
\ref{tab:comparisontot}. It is clear then that the adjustment outside the
calibration period is not the most optimal according to this method. We can
compare also these two models in terms of the estimated coefficients of the
external forcings in equation \eqref{eq:sswf}. The estimated density function
for each coefficient is shown in Figure \ref{fig:betas}. Note that the INLA's
estimates are more precise and their magnitude are quite similar for the Solar and Vulcanism
forcings. The greenhouse-gases coefficient is significantly smaller than the one
obtained in the MCMC model, mainly due to a larger effect of the remaining
reduced proxies on the reconstruction, but in general, this external forcings
represents the one with greatest weight when we describe the mean behavior of the anomalies.

In terms of computational efficiency the INLA procedure is quite remarkable, not only
for our case but in most of the previous works on the topic (see \cite{Rue2009},
\cite{Blangiardo2013}, \cite{Ruiz-Cardenas2012} for a few examples). The
computational time of an MCMC method was
aproximately 14 hours, whereas the computational time of INLA's best model with
8 RPs (most complex model) was aproximately 11 minutes \footnote{This comparison
was performed on an Ubuntu 16.04 server with Intel Xeon E5-2630 (8-cores,
2.40GHz) and 64 GB of RAM.}. 



% The same can be said
% for a closer look of the reconstructions during the period 1900-2000 (see
% . For comparison purposes,
% we also include the reconstructions of Model SSwF with the PCR method (see the
% upper panels of 
% figures ), where the
% main difference corresponds to the period 1-250 in terms of the variance of the
% mean function, mainly due to the sensitivity of the B-Spline basis to
% discrepancies among reduced proxies in a fixed time period. Also note that the
% resulting posterior densities of the external forcings (see Figure
% \ref{fig:betas} in the Appendix) suggest that both $S_t$
% and $\tilde C_t$ are quite significant to explain the temperature anomalies.    




% Finally, we also compare the nature of the linear function that determines the
% mean behavior of the state equation in both models. As the coefficients that
% accompany the covariates are random (External forcings for Model SSwF, B-Splines
% for Model SSnF) then it is possible to calculate confidence bands for the mean
% function in both cases. This is illustrated in Figure \ref{fig:meanfunction}.
% \begin{figure}
%   \centering
%   \includegraphics[scale=0.4]{MeanFunction_comp}
%   \caption{Mean function of the state equation and its 95\% confidence band.}
%   \label{fig:meanfunction}
% \end{figure}
% Note that the variance and the stability of the mean function are much greater
% in the best case of Model SSnF than in the best case of Model SSwF. This could be an
% indicator that Model SSwF has a greater bias in its mean behavior, which is a
% negative aspect in the reconstruction as a whole.




\section{Conclusions.}
\label{sec:conclusions}

%\printbibliography

\bibliographystyle{plain}
\bibliography{biblioteca}


\newpage
\appendix

\section{Additional Plots}

\begin{figure}[H]
  \centering
 \includegraphics[scale=0.38]{RPs_type} 
  \caption{Reduced Proxies among methods.}
  \label{fig:RPs}
\end{figure}


\begin{figure}[H]
  \centering
 \includegraphics[scale=0.38]{CorMatrixRPs} 
  \caption{Correlation Matrices for each Reduced Proxy among methods.}
  \label{fig:CorrRPs}
\end{figure}
% \begin{figure}[H]
%   \centering
%   \includegraphics[scale=0.4]{PAGES_composites_PCR5} 
%   \caption{Reduced Proxies using PC regression.}
%   \label{fig:proxiespcr}
% \end{figure}

\begin{figure}[H]
  \centering
  \includegraphics[scale=0.35]{forcings}
  \caption{Climate Forcings in the Common Era (1-2000 AD)}
  \label{fig:forcings}
\end{figure}

\begin{figure}[H]
  \centering
  \includegraphics[scale=0.45]{PlotBetas}
  \caption{Posterior densities of $\beta_1$, $\beta_2$ and $\beta_3$ for Model
    sPCR-SSwF and MCMC using PCR single reduced proxy.}
  \label{fig:betas}
\end{figure}

% \begin{figure}[H]
%   \centering
%   \includegraphics[scale=0.35]{RecCE_PCR_Splines}
%   \caption{Paleoclimate Reconstruction in the Common Era (CE) with 95\%
%     prediction bands. PCR Method - Model 2.}
%   \label{fig:paleoCE3}
% \end{figure}

% \begin{figure}[H]
%   \centering
%   \includegraphics[scale=0.4]{Rec1900_PCR_Splines}
%   \caption{Paleoclimate Reconstruction 1900-2000 with 95\%
%     prediction bands. PCR Method - Model 2.}
%   \label{fig:paleo19003}
% \end{figure}

\begin{figure}[H]
  \centering
  \includegraphics[scale=0.35]{RecCE_MCMC}
  \caption{Paleoclimate Reconstruction in the Common Era (CE) with 95\%
    prediction bands. Model sPCR-SSwF under two methods: INLA and MCMC.}
  \label{fig:paleoCE4}
\end{figure}

\begin{figure}[H]
  \centering
  \includegraphics[scale=0.35]{Rec1900_MCMC}
  \caption{Paleoclimate Reconstruction 1900-2000 with 95\%
    prediction bands. Model sPCR-SSwF under two methods: INLA and MCMC.}
  \label{fig:paleo19004}
\end{figure}

\begin{figure}[H]
  \centering
  \includegraphics[scale=0.35]{compMeans}
  \caption{Comparison of the distribution of reconstructed anomalies for different time horizons. Model sPCR-SSwF.}
  \label{fig:compmeans}
\end{figure}

\begin{figure}[H]
  \centering
  \includegraphics[scale=0.35]{compTrends}
  \caption{Comparison of the distribution of trends of reconstructed anomalies for different time horizons. Model sPCR-SSwF.}
  \label{fig:comptrends}
\end{figure}
\end{document}

\end{document}
