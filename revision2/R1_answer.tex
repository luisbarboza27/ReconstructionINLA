\documentclass[11pt]{article}

\usepackage{amsmath,amsfonts,amssymb,amsthm,verbatim,multirow,hyperref,subfig,footnote,graphicx,array,xr,booktabs}
\usepackage{times,natbib}
\usepackage[textheight=600pt]{geometry}
\usepackage[usenames,dvipsnames]{color}
%%%%%%%%%%%%%%
%\usepackage[spanish]{babel}
%\usepackage[utf8]{inputenc}
%%%%%%%%%%%%%



% \oddsidemargin 0pt
% \topmargin 0pt
% \headheight 0pt
% \headsep 0pt
% \textwidth 15cm  
% \textheight  24cm
% \footskip 1cm

\newcommand{\lb}[1]{\color{ForestGreen}\textbf{[Luis B.: #1]}\normalcolor}
\newcommand{\bl}[1]{\color{red}\textbf{[Bo: #1]}\normalcolor}
\newcommand{\jeg}[1]{\color{blue}\textbf{Julien: #1]}\normalcolor}

\begin{document}
\begin{center}
  {\Large \textbf{Response to the comments of the Referee.}}
\end{center}

We are grateful for the reviewer's careful commentary on the paper, which has improved
our presentation of the article. In what follows, we state the reviewer's
comments in italics, and describe our response in roman.

\begin{itemize}
  
\item \textit{The authors have presented a thorough revision and have greatly
    focused the scope and presentation of the paper. They explore different
    models for reducing dimension of the proxy dataset and explore the impact of
    different inference algorithms and the use of forcings to inform the
    reconstruction. I appreciate the authors’ effort in fitting the different
    models with the full and reduced proxy datasets which suggest there is only
    a small difference in predictive skill between the screened and unscreened
    data - I will always disagree whether this is an ideal way to proceed, but I
    recognize compromises need to be made to analyze the data. I also applaud
    the authors for the inclusion of code to reproduce the models presented. The
    ideas in the revised manuscript are more focused and organized and the thesis of the paper is more clear.
}

We thank the referee for this positive feedback. 
\item \textit{I still have some issues with the methodology and conclusions -
    primarily the evaluation of the reconstruction relative to another
    reconstruction - does this lead to group-think?}

It is true that we used the borehole reconstructions of PS04 as benchmark, but it is essential to note that these data sources are completely independent of the proxy data, and largely independent of surface temperature datasets (used only as baseline). There is thus no danger of group-think here, though both reconstructions have errors. We have amended the text in the conclusion and section 4.2 to make this point clearer.


  
\item \textit{I am also concerned with the
    difference in mean between the INLA and MCMC model in Figure (7) - are you
    running into issues of identifiability? Would a restricted regression model
    resolve this issue (Hanks et. al. (2015), Hughes and Haran (2013)) or is
    this due to the INLA model being a different model than the MCMC model (e.g.
    B14).}

We do not think there is identifiability issue with the parameters
\(\beta_1,\beta_2\) and \(\beta_3\) by examining the results in Figure 8. The plot shows that the parameter estimates for both INLA and MCMC are well estimated and the two sets of estimates are indeed comparable between the two model fitting approaches.
However, the mean behavior of the latent variable
\(T_t\) seems to be sensitive to the approximation of the
joint posterior distribution. \bl{the following conjecture may need more support} We
expect that as the reconstruction period increases more
differences among INLA and MCMC may be observed due to an increase in model complexity
making the overall approximation more difficult to attain. It is known that
complex and high dimensional MCMC schemes may suffer unfavorable convergence
properties (see \cite{Rajaratnam2015} for example).  
  
\item \textit{Figure 6b shows that the suite of models have a hard time identifying
    the mean response relative to the comparison reconstruction. Is this
    consistent across all of the models or only an issue for the 4 you
    presented. If this is a consistent issue, it raises questions about the
    real-world implications on the discussion of climate change if small changes
    in model assumptions can produce different predictions (see Schofield et.
    al. 2016). The paper would be improved with one or two sentences along the
    lines of ``Assumptions about the model choice have implications in the exact
    version of the prediction produced, resulting in different reconstructions.
    However; regardless of assumptions made in model choice, the rate of
    temperature increase in the last century relative to the past 2000 has been unprecedented.''}

\lb{Julien: Once we have the Figure 6b corrected we can answer this question.}  
  
\item \textit{line 17 page 1 - state-of-the-art proxy}

Thanks. We have corrected the typo.
  
\item \textit{page 3, line 14 Section vs. section}

  Thanks. We have corrected the typo.
  
\item \textit{page 7 lines 31-33}

We could not find any comment regarding this page. 

  
\item \textit{page 11, eq (4), what is \(m-=\) in \(\sum_{m-=1}\)?}

  Thanks. That was a typo. It should be \(\sum_{m=1}\). We have fixed that. 
  
\item \textit{page 12 line 56 - is precision the word you mean here, or is it
    better described as accuracy?}
  Yes, you are right. The correct word should be ``accuracy''. Thank you. 
\end{itemize}

\bibliographystyle{apalike}
\bibliography{biblioteca}


\end{document}


